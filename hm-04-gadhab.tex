% hm-04-gadhab.tex

\documentclass[xcolor=dvipsnames]{beamer}

\usepackage{graphicx}
% \usepackage{wrapfig}
% \usepackage{colortbl}
\usepackage{alltt}
% \definecolor{myblue}{rgb}{0.8,0.85,1}

\mode<presentation>
{
  \usetheme{Warsaw}
  \setbeamercovered{transparent}
}
% \usecolortheme[named=OliveGreen]{structure}
\setbeamertemplate{navigation symbols}{} 
\setbeamertemplate{blocks}[rounded][shadow=true] 

\newif\ifBCITCourse
\BCITCoursetrue
\BCITCoursefalse
\newif\ifWhichCourse
\WhichCoursetrue
% \WhichCoursefalse
\ifBCITCourse
\ifWhichCourse
\newcommand{\CourseName}{Statistics for Food Technology}
\newcommand{\CourseNumber}{MATH 1441}
\newcommand{\CourseInst}{BCIT}
\else
\newcommand{\CourseName}{Calculus for Geomatics}
\newcommand{\CourseNumber}{MATH 1511}
\newcommand{\CourseInst}{BCIT}
\fi
\else
\newcommand{\CourseName}{Philosophy and Literature}
\newcommand{\CourseNumber}{PHIL 375}
\newcommand{\CourseInst}{UBC}
\fi

\title{Gadamer and Habermas}
\subtitle{{\CourseNumber}, {\CourseInst}}

\author{\CourseName}

\date{October 24, 2017}

\begin{document}

\begin{frame}
  \titlepage
\end{frame}

\begin{frame}
  \frametitle{iClicker Question}
Choose from the following options. This item will not be graded.
\begin{block}{iClicker Question}
Which concept does Gadamer seek to rehabilitate as not being contrary
to reason and freedom?
\end{block}
\begin{description}
\item[A\hspace{.2in}$\blacktriangleright$] determinism
\item[B\hspace{.2in}$\blacktriangleright$] tradition
\item[C\hspace{.2in}$\blacktriangleright$] innate abilities
\item[D\hspace{.2in}$\blacktriangleright$] genius
\end{description}
\end{frame}

\begin{frame}
  \frametitle{iClicker Question}
Choose from the following options. This item will be graded.
\begin{block}{iClicker Question}
What is the role of time for understanding, according to Gadamer?
\end{block}
\begin{description}
\item[A\hspace{.2in}$\blacktriangleright$] it is a positive and productive condition that enables understanding
\item[B\hspace{.2in}$\blacktriangleright$] it is fundamentally asymmetric, you can go forwards but not backwards
\item[C\hspace{.2in}$\blacktriangleright$] it is a gulf to be bridged that separates us from historical events
\item[D\hspace{.2in}$\blacktriangleright$] it is scientifically reducible to archeological artefacts
\end{description}
\end{frame}

\begin{frame}
  \frametitle{iClicker Question}
Choose from the following options. This item will be graded.
\begin{block}{iClicker Question}
Which war is used as an example that the name we give it now could
not have been used at its outbreak?
\end{block}
\begin{description}
\item[A\hspace{.2in}$\blacktriangleright$] The Thirty Years War
\item[B\hspace{.2in}$\blacktriangleright$] The First World War
\item[C\hspace{.2in}$\blacktriangleright$] The Great War
\item[D\hspace{.2in}$\blacktriangleright$] The War of Roses
\end{description}
\end{frame}

\begin{frame}
  \frametitle{iClicker Question}
Choose from the following options. This item will be graded.
\begin{block}{iClicker Question}
Habermas makes reference to which philosopher's linguistic analysis?
\end{block}
\begin{description}
\item[A\hspace{.2in}$\blacktriangleright$] Russell
\item[B\hspace{.2in}$\blacktriangleright$] Putnam
\item[C\hspace{.2in}$\blacktriangleright$] Hegel
\item[D\hspace{.2in}$\blacktriangleright$] Wittgenstein
\end{description}
\end{frame}

\begin{frame}
  \frametitle{Gadamer's Question}
Here is the question that Gadamer wants answered:
\begin{quote}
  How can hermeneutics, once freed from the ontological obstructions
  of the scientific concept of objectivity, do justice to the
  historicity of understanding? (268)
\end{quote}
Note the difference between \emph{explanation} (which seeks to give
necessary and sufficient conditions for the reduction of a concept to
more primitive concepts) and \emph{understanding} (whose structure is
circular and non-reductive). For Heidegger, understanding
$\leftrightarrow$ circularity $\leftrightarrow$ temporality are
inseparable.
\end{frame}

\begin{frame}
  \frametitle{Being and Time}
  \begin{block}{Martin Heidegger}
    What is decisive is not to get out of the circle but to come into
    it the right way {\ldots} in the circle is hidden a positive
    possibility of the most primordial kind of knowing. To be sure, we
    genuinely take hold of this possibility only when, in our
    interpretation, we have understood that our first, last and
    constant task is never to allow our fore-having, fore-sight and
    fore-conception to be presented to us by fancies and popular
    conceptions, but rather to make the scientific theme secure by
    working out these fore-structures in terms of the things
    themselves. (\emph{Being and Time} 32:195)
  \end{block}
\end{frame}

\begin{frame}
  \frametitle{Martin Heidegger}
  ``Heidegger gave understanding an ontological orientation by
  interpreting it as an existential and when he interpreted Dasein's
  mode of being in terms of time'' (296). Time is not a gulf to be
  bridged, but it supports the ground for events to unfold (297)
  $\longrightarrow$ recognize temporal distance as a positive and
  productive condition enabling understanding! (For example, it is
  much more difficult to assess the historical significance of current
  events than of historical events.)
\end{frame}

\begin{frame}
  \frametitle{Hermeneutic Circle}
  Here are a few points about the hermeneutic circle, a concept that
  (I think) originates with Friedrich Schleiermacher.
  \begin{itemize}
  \item at first, the hermeneutic circle describes interpretation as
    moving back and forth from parts of a text to the whole of a text.
    Neither can be understood without the other. This is not a
    question-begging procedure, as it would be in science or logic
    (the bus is late because it arrived three minutes after the
    scheduled time)
  \item but then, circles appear in hermeneutics in many other places:
    for example, when you are trying to understand the historical
    past, you must also try to understand the present, for which it is
    essential to understand the historical past
  \end{itemize}
  \begin{quote}
    the circle of understanding is not a ``methodological'' circle,
    but describes an element of the ontological structure of
    understanding (294)
  \end{quote}
\end{frame}

\begin{frame}
  \frametitle{Hermeneutic Circle}
  \begin{itemize}
  \item another hermeneutic circle, according to Gadamer, is the way
    in which prejudices are essential for understanding and
    understanding is essential for discriminating between productive prejudices
    that enable understanding and prejudices that hinder it and lead
    to misunderstandings (295)
  \item ``Heidegger derives the circular structure of understanding
    from the temporality of Dasein'' (268) Heidegger writes, ``in the
    circle is hidden a positive possibility'' (269)
  \item the reason why circularity is virtuous, not vicious, for
    interpretation is because it is \emph{existential}, i.e.\ ``the
    circle possesses an ontologically positive significance'' (269)
  \end{itemize}
\end{frame}

\begin{frame}
  \frametitle{Prejudice}
  It is important to understand that for Gadamer, understanding of the
  text is already present as you read the text. You do not get to it
  as an afterthought. 
  \begin{quote}
    Working out this fore-projection, which is constantly revised in
    terms of what emerges as she penetrates into the meaning, is
    understanding what is there. (269)
  \end{quote}
  Objectivity refers only to the process, not to the outcome (compare
  this to Sartre's cowardice, which is the only objective moral
  failure). Inappropriate fore-meanings do not survive this
  process---as in Schechtman, when inappropriate narratives which
  violate the reality constraint do not hold up under interrogation
  (Gadamer on page 271: ``the hermeneutical task becomes of itself a
  questioning of things'').
\end{frame}

\begin{frame}
  \frametitle{Misunderstandings}
  There is some Popperian asymmetry detectable in Gadamer's analysis
  of misunderstanding. The reader must always be open to new meanings,
  just as Popper's scientist must always be open to new creative
  hypotheses. However,
  \begin{quote}
    {\ldots} meanings cannot be understood in an arbitrary way. Just
    as we cannot continually misunderstand the use of a word without
    its affecting the meaning of the whole, so we cannot stick blindly
    to our own fore-meaning about the thing if we want to understand
    the meaning of another. (271)
  \end{quote}
  Therefore, certain interpretive approaches and fore-meanings can be
  falsified. The harmony of all the details with the whole is the
  criterion of correct understanding (291).
\end{frame}

\begin{frame}
  \frametitle{Hermeneutics of Trust}
  For Gadamer, a hermeneutics of suspicion is always derivative of a
  hermeneutics of trust (the contrast between the two is due to Paul
  Ricoeur, who wrote about this after Gadamer published \emph{Truth
    and Method}). Gadamer says, ``only when this assumption proves
  mistaken [the assumption that the text has integrity]---i.e.\ the
  text is not intelligible---do we begin to suspect the text'' (294).
  Understanding is like speaking your native language: it is natural,
  constitutive of being. Misunderstanding is derivative and
  artificial, such as when someone speaks to you in a foreign language
  that you do not understand.
  \begin{quote}
    It is only when the attempt to accespt what is said as true fails
    that we try to ``understand'' the text, psychologically or
    historically, as another's opinion (294)
  \end{quote}
\end{frame}

\begin{frame}
  \frametitle{Lifeworld}
  Openness always includes situating the other meaning in relation to
  the whole of our own meanings or ourselves in relation to it (271).
  For example, Martin Luther's list on the next slide cannot and must
  not be interpreted without the lifeworld including our experience of
  the holocaust.
\end{frame}

\begin{frame}
  \frametitle{On the Jews and Their Lies}
  \begin{enumerate}
  \item to burn down Jewish synagogues and schools and warn people
    against them;
  \item to refuse to let Jews own houses among Christians;
  \item for Jewish religious writings to be taken away;
  \item for rabbis to be forbidden to preach;
  \item to offer no protection to Jews on highways;
  \item for usury to be prohibited and for all silver and gold to be
    removed, put aside for safekeeping, and given back to Jews who
    truly convert; and
  \item to give young, strong Jews flail, axe, spade, and spindle, and let them earn their bread in the sweat of their brow.[9]
  \end{enumerate}
\end{frame}

\begin{frame}
  \frametitle{Tradition}
  Heidegger, according to Gadamer, secures the scientific theme by
  framing it within the understanding of tradition (272). The
  fundamental prejudice of the Enlightenment is the prejudice against
  prejudice itself, which denies tradition its power (273). For the
  enlightenment, negative prejudice is distinguished from founded
  judgment by reference to justificatory method. The two sources of
  prejudice:
  \begin{itemize}
  \item human authority
  \item overhastiness
  \end{itemize}
\end{frame}

\begin{frame}
  \frametitle{Romanticism}
  The romantic reversal of the Enlightenment's criteria of value
  perpetuates the abstract contrast between myth and reason (275)
  \begin{itemize}
  \item the world of myth
  \item unreflective life
  \item not yet analyzed by consciousness
  \item close to nature
  \item Christian chivalry
  \item simplicity of peasant life
  \end{itemize}
\end{frame}

\begin{frame}
  \frametitle{Romanticism}
  The achievements of Romanticism (276):
  \begin{itemize}
  \item revival of the past
  \item discovery of the voices of the peoples in their songs
  \item collecting of fairy tales and legends
  \item cultivation of ancient customs
  \item discovery of worldviews implicit in languages
  \item study of religion and wisdom in India
  \item nineteenth-century historiography
  \end{itemize}
However, Romanticism sees itself as fulfillment of the Enlightenment,
achieving the kind of objectivity in historiography that was achieved
in the natural sciences.
\end{frame}

\begin{frame}
  \frametitle{Overcoming the Global Demand}
  \begin{block}{Hans-Georg Gadamer}
    The overcoming of all prejudices, this global demand of the
    enlightenment, will prove to be itself a prejudice, the removal of
    which opens the way to an appropriate understanding of our
    finitude which dominates not only our humanity but also our
    historical consciousness. (277)
  \end{block}
Gadamer says that our own prejudice is properly brought into play by
being put at risk (299).
\end{frame}

\begin{frame}
  \frametitle{Reason}
  Gadamer emphasizes that ``reason exists for us only in concrete,
  historical terms'' (277, Habermas calls this idea
  ``de-transcendentalized reason''). The fundamental epistemological
  question is not, what can I know detached from all experience; but:
  what is the ground of the legitimacy of prejudices? (278) Gadamer
  defends the role of authority. Gadamer is especially vocal about
  resisting the antagonism between reason and tradition/authority that
  has been built up by both the enlightenment and romanticism.
  \begin{quote}
    {\ldots} there is no such unconditional antithesis between
    tradition and reason {\ldots} the romantic faith in the ``growth
    of tradition,'' before which all reason must remain silent, is
    fundamentally like the enlightenment and just as prejudiced (282)
  \end{quote}
\end{frame}

\begin{frame}
  \frametitle{Authority}
  Authority is not based on abdication of reason but on an act of
  acknowledgment and knowledge (281). Granting authority to a superior
  is an act of freedom and reason. This applies to morality and
  supplies us with a critique of both existentialism and reason-based
  moral views (Kant, Mill):
  \begin{quote}
    The real force of morals, for example, is based on tradition. They
    are freely taken over but by no means created by a free insight or
    grounded on reasons. (282) Romanticism was wrong by opposing
    tradition and freedom.
  \end{quote}
\end{frame}

\begin{frame}
  \frametitle{Wirkungsgeschichtliches Bewusstsein}
  Gadamer wants to abolish the abstract antithesis between tradition
  and historical research, between history and the knowledge of it
  (283). Wirkungsgeschichte. Difference between the history of
  mathematics and the history of the human sciences. You don't need to
  read Max Planck's papers to understand quantum mechanics, but you
  would want to read ``Droysen and Mommsen'' (285). Historiography is
  not only research, it itself established tradition. It is
  distinguished from natural science by having as its object the
  subject conducting the inquiry. ``The true historical object is not
  an object at all, but the unity of the one and the other'' (299).
  Understanding is, essentially, a historically effected event.
\end{frame}

\begin{frame}
  \frametitle{Open Grammars}
  There is a school of thought that says that our thinking is
  constrained by language depending on the particular language
  (Sapir-Whorf hypothesis).
  \begin{itemize}
  \item colour terms
  \item Pirah{\~a}: lack of number terms
  \item ``empty'' barrels that contained explosive vapours
  \item Chinese and counterfactuals
  \item Swedish prepositions and Finnish cases
  \end{itemize}
  For nativist objections to linguistic determinism see Steven
  Pinker's book \emph{The Language Instinct}, chapter ``Mentalese.''
  \begin{quote}
    First, let us do away with the folklore that parents teach their
    children language. (The Language Instinct, 28)
  \end{quote}
\end{frame}

\begin{frame}
  \frametitle{Open Grammars}
  For J{\"u}rgen Habermas, linguistic determinism is undermined by
  language itself. Language rules (which Habermas calls a grammar)
  enable transcendance.
  \begin{quote}
    every ordinary-language grammar opens up the possibility of also
    transcending the language that it establishes (143)
  \end{quote}
  \begin{quote}
    we are never enclosed within a single grammar {\ldots} the first
    grammar that one masters also enables one to step outside it and
    interpret something foreign, to make something that is
    incomprehensible intelligible (143)
  \end{quote}
  \begin{quote}
    the language games of the young do not simply reproduce the
    practice of the old (148; the impossibility of double-speak)
  \end{quote}
  Reason is always bound up with language and always beyond the
  languages in which it is expressed (144).
\end{frame}

\begin{frame}
  \frametitle{Wittgenstein}
  Wittgenstein's \alert{language games}. Wittgenstein is the
  detranscendentalizer of language, as Habermas is the
  detranscendentalizer of reason. Wittgenstein's language games resist
  positivism's formalized languages (for example, Gottlob Frege's
  \emph{Begriffsschrift}).

\bigskip

  For Wittgenstein, understanding is recapitulation of the training
  through which native speakers are socialized into their form of
  life. For Gadamer, this is not sufficient. Hermeneutics kicks in
  when the language game becomes problematic. Interpreters are like
  translators.
\end{frame}

\begin{frame}
  \frametitle{Wittgenstein}
  \begin{tabular}{|l|l|l|}\hline
    & Wittgenstein & Habermas \\ \hline
    language & situated & situated \\ \hline
    language & monadological & dialectical \\ \hline
  \end{tabular}
  \begin{block}{J{\"u}rgen Habermas}
    The lifeworlds established by the grammar of language games are
    not closed life forms, as Wittgenstein's monadological conception
    suggests {\ldots} Wittgenstein failed to recognize that the same
    rules also include the conditions of the possibility of
    interpretation. (147)
  \end{block}
  Dialectic, rather than monadology, leads to the possibility of
  revision and elastic renewal (148) (``language spheres are not
  monadically sealed but porous,'' 149).
\end{frame}

\begin{frame}
  \frametitle{Fusing Horizons}
  ``Hermeneutic understanding, which is only articulated in situations
  of disturbed consensus, is as fundamental to the understanding of
  language as is primary consensus'' (148).

\bigskip

Wittgenstein resisted positivism by introducing the practice of
language games and Husserl's concept of lifeworld to linguistics. He
did not go far enough by viewing language ahistorically and in terms
of the reproduction of fixed patterns (149). The contrast with formal
languages (communications, not derivations). The speech situation is
not a model for a formalized language (150).

\bigskip

  Understanding does not rest on empathy (as it did for Dilthey), but
  on ``the attainment of a higher universality'' (Gadamer, 151).
  ``Understanding is always the fusion of horizons.'' 
\end{frame}

\begin{frame}
  \frametitle{Hermeneutic Circle}
  \begin{itemize}
  \item circular relationship of pre-understanding and the explication
    of what is understood---we can decipher the parts of a text only
    if we anticipate an understanding (152)
  \item the circle is neither subjective nor objective, but
    characterized by the movement of interpreter and tradition
  \item it is not an act of subjectivity because it proceeds from a
    common bond with tradition; this bond, however, is constantly
    being developed
  \item we have always already understood the tradition (Socrates'
    maieutics)
  \item hermeneutic understanding is the interpretation of texts with
    the knowledge of texts that have already been understood (153)
  \item the possible objectivity of experience is endangered by the
    illusion of objectivity
  \end{itemize}
\end{frame}

\begin{frame}
  \frametitle{Wirkungsgeschichtliches Bewusstsein}
  \begin{block}{Hans-Georg Gadamer}
    The actual meaning of the text, as it speaks to the interpreter,
    does not depend on the contingencies of the author and those whom
    he originally wrote for. At least it is not exhausted by them, for
    it is always partly determined also by the historical situation
    {\ldots} the meaning of a text goes beyond its author
  \end{block}
Arthur Danto, an American art critic, introduces the significance of
narrative to Gadamer's idea of historical-effect consciousness.
Narratives stand in contrast to what neo-positivists have called
protocol sentences. The latter are
\begin{itemize}
\item neutral with respect to the time of occurrence
\item narrative elements are irrelevant in the framing
\end{itemize}
Danto's machine or the ideal chronicler---worthless for
historiography? (Comey's memos.)
\end{frame}

\begin{frame}
  \frametitle{History of the Present}
  Retrospective projections:
  \begin{quote}
    The meaning that thus accrues to the events retrospectively
    emerges only in terms of the schema of possible action, namely, as
    if the meaning, incorporating the knowledge of those born later,
    had been intended. (158)
  \end{quote}
This may be problematic. Often what fascinates us about history is
that we get to watch actors and empathize with them precisely in the
sense that we want to understand their motives, desires, and beliefs
before their intentions were either clear or fulfilled/thwarted.
Examples: Gauguin, Anna Karenina, Brunhilde Pomsel. See Milan
Kundera's essay ``Paths in the Fog.''
\end{frame}

\begin{frame}
  \frametitle{Transcendental Versus Situated Reason}
The self-transcendence inherent in language opens the possibility for
transitioning between languages. Reason, however, does not transcend
languages---it is situated in and between them. For Habermas, reason
always emerges from people talking to each other. There is no such
thing as Kant's `pure reason.'   
\end{frame}

\begin{frame}
  \frametitle{Communicative Rationality}
Danto emphasizes that history is of a piece and not divisible into
pure descriptions and interpretations (159). G{\"o}del's
incompleteness theorems: a formal axiom system containing arithmetic
can be complete, it can be consistent, but it cannot be both; and its
consistency cannot be demonstrated from within the system.

\bigskip

Habermas seeks to expand on enlightenment principles (instead of
rejecting them) by proposing a framework for \alert{communicative
  rationality}. Communicative rationality rests on an ideal speech
situation in which discourse is immunized against repression and
inequality. Here are some rules (Habermas calls them presuppositions):
\end{frame}

\begin{frame}
  \frametitle{Communicative Rationality}
\begin{itemize}
\item participants in communicative exchange are using the same
  linguistic expressions in the same way
\item no relevant argument is suppressed or excluded by the
  participants
\item no force except that of the better argument is exerted
\item all the participants are motivated only by a concern for the
  better argument
\item everyone capable of speech and action is entitled to
  participate, and everyone is equally entitled to introduce new
  topics or express attitudes, needs, or desires
\item no validity claim is exempt in principle from critical
  evaluation in argumentation
\end{itemize}
\end{frame}

\begin{frame}
  \frametitle{Aristotle's Techne, Science, and Praxis}
  Habermas's three moments of hermeneutic knowledge which it has in
  common with Aristotle's political-ethical knowledge:
  \begin{itemize}
  \item reflexive: it is always also self-knowledge and shapes who
    we are
  \item internalized: practical knowledge continues a process of
    socialization (Polyphemus)
  \item global: it is never restricted to particular parts of the
    lifeworld (the ability to operate a television vs the ability to
    raise a child)
  \end{itemize}
  Technical rules (fundamental predicates, invariant rules of
  application) vs practical rules (consensus, intersubjectivity,
  pre-understanding) (165).
\end{frame}

\begin{frame}
  \frametitle{Habermas's Critique of Gadamer}
  \begin{itemize}
  \item Gadamer's agreement with the positivists that hermeneutics
    transcends science (167)
  \item the confrontation of truth and method should not have led
    Gadamer to an abstract opposition between hermeneutic experience
    and methodical knowledge
  \item shifting the balance of authority and reason: Gadamer fails to
    recognize the power of reflection that unfolds in \emph{Verstehen}
    (168)
  \item Gadamer's absolutization of hermeneutics (169)---this is in
    contradiction to a rule of communicative action
  \item the opposition between the legitimacy of prejudices validated
    by tradition and the power of reflection, which can reject
    tradition (170)
  \end{itemize}
\end{frame}

\end{document}
