% hm-07-skepticism.tex

\documentclass[xcolor=dvipsnames]{beamer}

\usepackage{graphicx}
% \usepackage{wrapfig}
% \usepackage{colortbl}
\usepackage{alltt}
% \definecolor{myblue}{rgb}{0.8,0.85,1}

\mode<presentation>
{
  \usetheme{Warsaw}
  \setbeamercovered{transparent}
}
% \usecolortheme[named=OliveGreen]{structure}
\setbeamertemplate{navigation symbols}{} 
\setbeamertemplate{blocks}[rounded][shadow=true] 

\newif\ifBCITCourse
\BCITCoursetrue
\BCITCoursefalse
\newif\ifWhichCourse
\WhichCoursetrue
% \WhichCoursefalse
\ifBCITCourse
\ifWhichCourse
\newcommand{\CourseName}{Statistics for Food Technology}
\newcommand{\CourseNumber}{MATH 1441}
\newcommand{\CourseInst}{BCIT}
\else
\newcommand{\CourseName}{Calculus for Geomatics}
\newcommand{\CourseNumber}{MATH 1511}
\newcommand{\CourseInst}{BCIT}
\fi
\else
\newcommand{\CourseName}{Philosophy and Literature}
\newcommand{\CourseNumber}{PHIL 375}
\newcommand{\CourseInst}{UBC}
\fi

\title{Skepticism}
\subtitle{{\CourseNumber}, {\CourseInst}}

\author{\CourseName}

\date{November 21, 2017}

\begin{document}

\begin{frame}
  \titlepage
\end{frame}

\begin{frame}
  \frametitle{iClicker Question}
Choose from the following options. This item will not be graded.
\begin{block}{iClicker Question}
The X argument: someone whose life is guided by usual practices is in
direct danger of becoming an X. Who is X?
\end{block}
\begin{description}
\item[A\hspace{.2in}$\blacktriangleright$] Oblomov
\item[B\hspace{.2in}$\blacktriangleright$] Eichmann
\item[C\hspace{.2in}$\blacktriangleright$] The Big Lebowski
\item[D\hspace{.2in}$\blacktriangleright$] Charles Duhigg
\end{description}
\end{frame}

\begin{frame}
  \frametitle{iClicker Question}
Choose from the following options. This item will be graded.
\begin{block}{iClicker Question}
What is the relevant pair of contrast for Odo Marquard's essay?
\end{block}
\begin{description}
\item[A\hspace{.2in}$\blacktriangleright$] accidents and narrative
\item[B\hspace{.2in}$\blacktriangleright$] accidents and coincidence
\item[C\hspace{.2in}$\blacktriangleright$] accidents and apriority
\item[D\hspace{.2in}$\blacktriangleright$] accidents and choices
\end{description}
\end{frame}

\begin{frame}
  \frametitle{The Necessity of the Accidental}
  Marquard argues that an element of the accidental is necessary for
  being human: 
  \begin{block}{Odo Marquard}
    To get rid of what is accidental [Hegel] would mean {\ldots} to
    get rid of philosophers {\ldots} without philosophers there would
    be no philosophy {\ldots} one would philosophy, in the name of
    philosophy, of philosophy (109)
  \end{block}
Later on, Marquard restates that one would ``rid man, in the name of
man, of man.''
\end{frame}

\begin{frame}
  \frametitle{Four Sections}
  \begin{enumerate}
  \item the program of making man absolute
  \item unavoidability of usual practices
  \item human beings are always more accidents than choice
  \item human freedom depends on the separation of powers
  \end{enumerate}
\end{frame}

\begin{frame}
  \frametitle{What Is the Accidental}
  The accidental emerges from an interesting property of the world as
  phenomenon (lifeworld, 121): that things could have been otherwise
  (Leszek Ko{\l{}}akowski: ``quod tamen potest esse aliter''). The
  accidental is what is neither impossible nor necessary. Distinguish
  between the
  \begin{itemize}
  \item arbitrarily accidental
  \item fatefully accidental (Nina Lugovskaya's strabismus)
  \end{itemize}
Si contingens, unde necessarium et vice versa. Again, we are
confronted with what looks like an irreducible dualism. Sartre,
following Kant: human beings are, without exception, not their
accidents, but only and completely their choice (contra ``moral
luck''). 
\end{frame}

\begin{frame}
  \frametitle{Absolutizing Ideologies}
  \begin{itemize}
  \item religion
  \item Descartes' methodical doubt: in dubito contra traditionem
    (when in doubt, against tradition)
  \item German idealism
  \item Marxism
  \item Habermasian discourse ethics
  \end{itemize}
Note how Marquard takes Gadamer's side in the debate over tradition;
and Popper's side in the debate over dogmatism/determinism/historicism
(open society). 
\end{frame}

\begin{frame}
  \frametitle{Marquard and Popper}
  Here are some similarities between Marquard and Popper:
  \begin{itemize}
  \item both were converted from adherence to abolutist ideologies
    (Popper from Marxism; Marquard from National Socialism)
  \item both assign an important role to creativity and the accidental
    in epistemology
  \item both vigorously criticize historicism (the idea that there is
    an intention in history beyond the intention of individuals which
    determines history)
  \item both emphasize the fallibility of knowledge
  \end{itemize}
\end{frame}

\begin{frame}
  \frametitle{Questions}
  \begin{itemize}
  \item ``human life is too short for executing the program of making
    man absolute'' (115)---compare and contrast the solution of German
    idealism/Marxism for this problem and Marquard's
  \item what is the difference between an epistemological skeptic like
    Descartes and an anti-dogmatic skeptic like
    Marquard/Popper/Ko{\l{}}akowski?
  \item note Marquard's unease with environmentalism: environmentalism
    has as one of its premises the ``eternal significance'' of human
    action, and our existing usual practices are often in conflict
    with this premise
  \item what is the Eichmann argument and how does Marquard address
    it?
  \end{itemize}
\end{frame}

\begin{frame}
  \frametitle{Questions}
  \begin{itemize}
  \item ``critique is, above all, conflict between usual
    practices'' (118)---what echoes of Kant do you hear here, and also
    a few lines later when Marquard rips into ``perfectionistic
    deontological demands''
  \item what kind of narrativist is Marquard? notice the difference
    between Aristotelian narratives (where form is imposed on content)
    and Marquardian narratives (where the contingent is necessary)
  \item note also similarities with Foucault: the significance of the
    contingent (versus Hume's necessity in the genealogy of morality,
    for example)
  \item note also similarities with Taylor: Marquard's emphasis on
    usual practices and tradition, which only make re-evaluation
    meaningful
  \end{itemize}
\end{frame}

\begin{frame}
  \frametitle{Questions}
  \begin{itemize}
  \item both Taylor and Marquard have a cautiously positive
    relationship with art and religion as human responses to mitigate
    arbitrariness
  \item skepticism (``Zweifel''): isosthenes diaphonia (in our course,
    for example, between the hermeneutic method and the scientific
    method)
  \item note Marquard's resistance to Mill's idea that there must be
    one principle of morality: ``there is never only one such power at
    work, but always a number of them'' (also for convictions,
    traditions, stories, souls, gods, points of orientation,
    freedoms---Marquard is a professing polytheist)
  \item finally, a nod to Jacques Derrida: ``what makes a human being
    free is not zero determination---the absence of all
    determinants---or the superior force of a single determinant, but
    a superabundance of determinants'' (124)
  \end{itemize}
\end{frame}

\begin{frame}
  \frametitle{What Is Enlightenment?}
  Human reason, says Marquard again with a bit of cheek directed at
  Immanuel Kant, is \alert{the abandonment of the effort to remain
    stupid}.
\end{frame}

\begin{frame}
  \frametitle{iClicker Question}
Choose from the following options. This item will be graded.
\begin{block}{iClicker Question}
What is Kolakowski's thesis statement?
\end{block}
\begin{description}
\item[A\hspace{.2in}$\blacktriangleright$] No explanatory method exists in the study of cultural history
\item[B\hspace{.2in}$\blacktriangleright$] facts can be abolished by a supreme synthesis
\item[C\hspace{.2in}$\blacktriangleright$] History would have turned out the same if certain important people (Descartes, Nietzsche) had died in infancy
\item[D\hspace{.2in}$\blacktriangleright$] Western civilization must decline as a matter of historical necessity
\end{description}
\end{frame}

\begin{frame}
  \frametitle{iClicker Question}
Choose from the following options. This item will be graded.
\begin{block}{iClicker Question}
According to Kolakowski, given the past, the future is
\end{block}
\begin{description}
\item[A\hspace{.2in}$\blacktriangleright$] unpredictable
\item[B\hspace{.2in}$\blacktriangleright$] directed at a specific end
\item[C\hspace{.2in}$\blacktriangleright$] predictable in its major movements
\item[D\hspace{.2in}$\blacktriangleright$] predictable in detail
\end{description}
\end{frame}

\begin{frame}
  \frametitle{iClicker Question}
Choose from the following options. This item will be graded.
\begin{block}{iClicker Question}
``The whole world is a work of art'' -- who is quoted saying this in Solnit?
\end{block}
\begin{description}
\item[A\hspace{.2in}$\blacktriangleright$] Virginia Woolf
\item[B\hspace{.2in}$\blacktriangleright$] Jean-Paul Sartre
\item[C\hspace{.2in}$\blacktriangleright$] Andrew Warhol
\item[D\hspace{.2in}$\blacktriangleright$] Immanuel Kant
\end{description}
\end{frame}

\begin{frame}
  \frametitle{iClicker Question}
Choose from the following options. This item will be graded.
\begin{block}{iClicker Question}
Why did Charlie Musselwhite stop drinking himself to death?
\end{block}
\begin{description}
\item[A\hspace{.2in}$\blacktriangleright$] because of a religious
  conversion experience
\item[B\hspace{.2in}$\blacktriangleright$] because he heard of a
  not-yet two-year old being saved from a well
\item[C\hspace{.2in}$\blacktriangleright$] because Aristotle
  appeared to him in a dream
\item[D\hspace{.2in}$\blacktriangleright$] because he was shipwrecked
  and had no access to alcohol
\end{description}
\end{frame}

\begin{frame}
  \frametitle{Student Questions}
  \begin{itemize}
  \item Why hermeneutics matter? What would be problematic if we don't
    study the impact of it? Of those espousing hermeneutics, which has
    the most sound logic for you? (Same for poststructural.) Is
    synthesis possible or are the diametrically opposed?
  \item Can you please give a brief explanation as to what Plato means
    by Truth and why Nietzsche has a problem with it?
  \item Is existentialism supporting free will?
  \end{itemize}
\end{frame}

\begin{frame}
  \frametitle{Student Questions}
  \begin{itemize}
  \item What is the difference between hermeneutics of trust and
    hermeneutics of tradition?
  \item What would Hume say about Parfit's reductionist view on
    personhood?
  \item I didn't understand to well what Sartre said about the human
    condition and what he says right after.
  \end{itemize}
\end{frame}

\begin{frame}
  \frametitle{Student Questions}
  \begin{itemize}
  \item How different are Foucault's ideas from those of Marx?
  \item Can we get short summaries of what a philosopher says in this
    respective paper for the ones that are hard to understand?
  \item On the topic of spatio-temporal blending of horizons. Does
    anyone take into account the personal, subjective interpretation
    of each person? We take the widely-held interpretation of the past
    to be the historical horizon, which we use to interpret something
    new. But surely not everyone in the past shares these beliefs.
    Same can be said about the spatial horizon. Do we take the
    subjective point of view into account when interpreting something?
  \end{itemize}
\end{frame}

\begin{frame}
  \frametitle{Student Questions}
  \begin{itemize}
  \item What is the hermeneutic circle?
  \item Would it be safe to say that all philosophical argument are
    sound in interpretation/explanation, until they are at their
    application? As in, the application in real life isn't as accurate
    as they are in theory?
  \item What does philosophy mean to you? And what do you think the
    philosophy mean in current times?
  \end{itemize}
\end{frame}

\begin{frame}
  \frametitle{Student Questions}
  \begin{itemize}
  \item Can you break down our readings into short topics so that we
    can better digest the information and study for our exam?
  \item What, according to Parfit, would happen if humans became aware
    that self-identity is an illusion or to some degree? And how would
    it be possible to be aware that the self is non-existing? From
    paradox, as Parfit agrees---is this somewhat impossible to
    explain?
  \item What is the significance of knowing your identity if the
    qualifications (such as contains half of the brain, or
    psychological continuity) can be so arbitrary? Do we have to know
    those answers?
  \end{itemize}
\end{frame}

\begin{frame}
  \frametitle{Student Questions}
  \begin{itemize}
  \item On the topic of narratives, do you believe/have you created a
    narrative for yourself?
  \item Why did the author in existential writing say thinking is a
    luxury?
  \item Hermeneutics of trust, even though it's characteristic of
    structuralism, isn't it more prevalent among people even today?
  \end{itemize}
\end{frame}

\begin{frame}
  \frametitle{Student Questions}
  \begin{itemize}
  \item You said that part of Foucault's ``Body of the Condemned'' was
    one of the best things you've ever read. I'm guessing it was the
    last pages but I don't see why; I think there's not enough
    explanation there; so maybe I'm missing something?
  \item In some of the older readings, I'm wondering if the ideas of
    truth and freedom are accessible/attainable/were able to be
    demonstrated to be inherent in women?
  \item What is more destructive: discipline and punish targeting the
    body or the soul?
  \end{itemize}
\end{frame}

\begin{frame}
  \frametitle{Student Questions}
  \begin{itemize}
  \item Do all readings we've done relate to hermeneutics? If so, how
    do some fit in which seem less clearly related, like Charles
    Hodges's On Method, or Charles Taylor's Foucault on Freedom and
    Truth? Or even Foucault's The Incitement to Discourse or The Body
    of the Condemned? Or Karl Popper's The Logic of Scientific
    Discovery?
  \item Are people with second-order desires really self-involved and
    selfish?
  \item And extremely general question: at this stage what should a
    student do if they feel rather lost in the course and have little
    to no idea whether they have grasped the concepts right/on the
    right track?
  \end{itemize}
\end{frame}

\begin{frame}
  \frametitle{Student Questions}
  \begin{itemize}
  \item How does self-narrative relate to Foucault?
  \item What is the deductive method? Is it the same as
    deconstruction?
  \item How do topics such as punishment, power tie back in with
    hermeneutics? The link between our course title in the readings is
    sometimes unclear.
  \end{itemize}
\end{frame}

\begin{frame}
  \frametitle{Student Questions}
  \begin{itemize}
  \item Foucault's idea of modern power relation makes us better or
    worse?
  \item Do philosophers ever agree on an answer? Or does everyone just
    keep adding new points of view to all topics? How do some authors
    claim that ``the enlightenment won''?
  \item What exactly does Charles Taylor mean by ``purposefulness
    without purpose''?
  \end{itemize}
\end{frame}

\begin{frame}
  \frametitle{Student Questions}
  \begin{itemize}
  \item Does Derek Parfit reject Marya Schechtman?
  \item When does knowledge become an all-trumping power?
  \item What would be Parfit's response to MS's claim that his view
    produces people who don't make plans, take responsibility for
    present or past?
  \end{itemize}
\end{frame}

\begin{frame}
  \frametitle{Student Questions}
  \begin{itemize}
  \item What does Gadamer mean by ``fusion of horizons''?
  \item Why do we use hypothetical questions that are
    unrealistic/could never happen?
  \item Is the power relation of Foucault a phenomenon or an
    ontological being? Does it make sense to say it's an ontological
    being?
  \end{itemize}
\end{frame}

\begin{frame}
  \frametitle{Student Questions}
  \begin{itemize}
  \item About power of knowledge, is this reflecting historical
    background, say the colonial era?
  \item Does Foucault rejects the idea of object? Given that Charles
    Taylor says his idea is Nietzschean.
  \item There was a discussion on the possible role of science in
    determining morals, something that was tentatively concluded to be
    impossible. Sam Harris fleshes out an position that it is possible
    in his book The Moral Landscape, using the empirical well-being of
    sentient creatures as his foundation. Is this position untenable
    purely on the assumption that empirical well-being is a good
    thing? Would it be possible to value well-being while finding his
    position unconvincing?
  \end{itemize}
\end{frame}

\begin{frame}
  \frametitle{Student Questions}
  \begin{itemize}
  \item How much of an extent do you believe these philosophers carry
    weight to the modern day? Are they only relevant because some of
    the issues are they still prevalent, or are they a product of
    their time and only matter in their time period.
  \item Can you go over what once again about the paper due end of
    term?
  \item Could you explain how you could defend the non-narrativist
    view against the narrativist view?
  \end{itemize}
\end{frame}

\begin{frame}
  \frametitle{Student Questions}
  \begin{itemize}
  \item What is value neutrality?
  \item When arguing against a certain author's stance, should we
    invite the narrative of another philosopher, or should we include
    our own voices? Take Taylor as an example, or whoever that was
    that wrote something about existentialism, since he was attacking
    what those who go against existentialism are saying, do we argue
    in light of his objections and answer with the voices of existing
    philosophers, or do we use our own voices? (For whatever context,
    essay or exam or philosophy in general.)
  \item Is pragmatism possible? Why and why not? Van Hoozer---I have
    no idea what his stance on the three parables are.
  \end{itemize}
\end{frame}

\begin{frame}
  \frametitle{Student Questions}
  \begin{itemize}
  \item What modern philosophers/philosophical schools of thought
    totally reject hermeneutics? If so who are they and why?
  \item What is so wrong with the hermeneutics of suspicion?
  \item Doesn't Sartre's article speak about a role for collective
    reflection, a social aspect to defining purpose, goals, morals,
    etc.? It seemed like you said there wasn't any ... (Though I may
    be mistaken about that ...)
  \end{itemize}
\end{frame}

\begin{frame}
  \frametitle{Student Questions}
  \begin{itemize}
  \item Why have we been historically taught to repress our sexual
    nature? Who said that sexual liberation was bad for society?
  \item How to Kant and Nietzsche differ when it comes to personal
    identity?
  \item Even though the logical positivists felt that some topics were
    impossible to talk about, did they still feel that
    things/questions such as ethics, aesthetics, where do we go when
    we die etc. still held importance in our lives? Even if they were
    logically only nonsense, they still hold importance. I want to
    know just how important they felt these things to be.
  \end{itemize}
\end{frame}

\begin{frame}
  \frametitle{Student Questions}
  \begin{itemize}
  \item What is hermeneutics?
  \item If hermeneutics is the interpretation of the text, doesn't
    that necessitate an ``interpreter'' or a ``self'' to do the
    interpreting?
  \item What does Van Hoozer say again?
  \end{itemize}
\end{frame}

\begin{frame}
  \frametitle{Student Questions}
  \begin{itemize}
  \item Can you please give a quick overview/refresher on Galen
    Strawson's views?
  \item What was Popper's main point in what he said about everything
    having probability zero?
  \item Is there such a thing as true altruism? Can it be viewed as an
    evolutionary tool to encourage cooperation/mutual support? Why do
    we ``feel good'' or get a ``warm fuzzy feeling''?
  \end{itemize}
\end{frame}

\begin{frame}
  \frametitle{Student Questions}
  \begin{itemize}
  \item Regarding the discussion from last week about Foucault and
    oppression: if we are all oppressed then wouldn't that be a given
    of life, hence, true oppression or freedom would be relative to
    the oppression imposed on us as a fact of life?
  \item Can only people that believe in control over their life
    narratives assert control/make changes?
  \item I am wondering if the use of technology, for example social
    networking sites such as Instagram or Facebook, help us to
    establish or maintain narratives.
  \end{itemize}
\end{frame}

\begin{frame}
  \frametitle{Student Questions}
  \begin{itemize}
  \item What's the best way to argue that Carnap's linguistic
    framework/principle of tolerance supports Habermas's passion for
    methodology and hermeneutics compared to Gadamer's theory that
    there is no methodology to hermeneutics?
  \item Why did you choose Foucault for this class? What's your
    opinion of Foucault's academic achievements?
  \item What does Nietzsche mean when he says only the "think" in "I
    think therefore I am" exists and "I" is only the creation of
    grammar? Does it make sense to say "deed without subject"?
  \end{itemize}
\end{frame}

\begin{frame}
  \frametitle{Student Questions}
  \begin{itemize}
  \item Will there be any re--weighting of the grades at the end of
    the course? For example, if someone does significantly better on
    the final paper than on the final exam, will the final exam be
    weighted less? I think philosophy requires a lot of heavy
    rumination on the topics at hand, so it surprised me that this
    course would have a final exam.
  \item How the power determines the forms and domains of knowledge?
  \item Isn't ``I think, therefore I am'' the same as ``which came
    first: the chicken/egg''?
  \end{itemize}
\end{frame}

\begin{frame}
  \frametitle{Student Questions}
  \begin{itemize}
  \item In your personal opinion, which has more weight: Foucault's
    theory or Taylor's rebuttal?
  \item What are intrinsic reasons?
  \item Do you think there is some kind of philosophy in ancient
    China? Or just agree with Hegel: there is no philosophy in ancient
    China.
  \end{itemize}
\end{frame}

\begin{frame}
  \frametitle{Student Questions}
  \begin{itemize}
  \item Parfit---also don't understand much.
  \item Is it possible to submit a rough draft of the essay before the
    deadline and get reviewed?
  \item Would you choose to keep or abolish the law that forbids
    having sex with children?
  \end{itemize}
\end{frame}

\begin{frame}
  \frametitle{Student Questions}
  \begin{itemize}
  \item What exactly do we have to know about Gadamer? He is confusing
    to understand. Same with Habermas.
  \item According to Popper, how is science and philosophy different?
  \item How would the personal view of a structuralist about a death
    differ from a poststructuralist?
  \end{itemize}
\end{frame}

\begin{frame}
  \frametitle{Student Questions}
  \begin{itemize}
  \item What are the key arguments undermining determinism?
  \item How can one argue against against Habermas's methodology when
    he doesn't provide any method for hermeneutics?
  \item Plutarch and Marya Schechtman, what do they think about a good
    meaningful narrativist life?
  \end{itemize}
\end{frame}

\begin{frame}
  \frametitle{Student Questions}
  \begin{itemize}
  \item Is foundationalism compatible with non-reductionism? Are they
    the same thing?
  \item On effective history, is all history effective? If not what
    distinguishes something that isn't and how do we know without
    knowledge of the future? (or all events)
  \item In the reductionist view, what separates humans from
    non-humans?
  \end{itemize}
\end{frame}

\begin{frame}
  \frametitle{Student Questions}
  \begin{itemize}
  \item To what extent does Taylor accurately represent Foucault?
  \item Why doesn't Sartre accept for moral guidance outside
    ourselves?
  \item What is the relation between strong evaluation and second
    order desires?
  \end{itemize}
\end{frame}

\begin{frame}
  \frametitle{Student Questions}
  \begin{itemize}
  \item Is modern societies' reliance on statistics inherently
    utilitarian in nature? How can Foucault's theories be anything
    resembling truth if truth is dependent on power and domination? If
    there's no objective truth, are Foucault's ideas only relevant to
    our current society?
  \item Are the contrasts between Habermas's and Gadamer's concepts of
    hermeneutics present in real life institutions? Like health field,
    government, science ...
  \item Optional: can you do a brief overview of every reading on one
    of the lectures in the final week?
  \end{itemize}
\end{frame}

\begin{frame}
  \frametitle{Student Questions}
  \begin{itemize}
  \item How can you truly know if your desires are yours and are
    outside of some external social structure?
  \item Dostoyevsky's look on letting go of rationality really
    possible?
  \item what is rational reconstruction?
  \end{itemize}
\end{frame}

\begin{frame}
  \frametitle{Student Questions}
  \begin{itemize}
  \item The readings we do for the course do not date far back, but
    why don't we look at more contemporary readings? Don't the
    concepts and ideas seem more falsifiable in relation to
    circumstances today?
  \item Does the philosopher only focus on men?
  \item Hume---can we summarize quick---don't understand much.
  \end{itemize}
\end{frame}

\begin{frame}
  \frametitle{Student Questions}
  \begin{itemize}
  \item Our class often reads and discusses both sides of a
    topic/view. Yet, we rarely sit down and consolidate that into a
    single view/choice. Personally, I am not sure how to begin
    weighing all of the sound (?) arguments out.
  \item Is Galen Strawson absolutely certain his views against
    narrativity won't change?
  \item What's up with Hume saying emotions are more important than
    reason; that reason should be a slave to emotion?
  \end{itemize}
\end{frame}

\end{document}
