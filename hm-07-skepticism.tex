% hm-06-skepticism.tex

\documentclass[xcolor=dvipsnames]{beamer}

\usepackage{graphicx}
% \usepackage{wrapfig}
% \usepackage{colortbl}
\usepackage{alltt}
% \definecolor{myblue}{rgb}{0.8,0.85,1}

\mode<presentation>
{
  \usetheme{Warsaw}
  \setbeamercovered{transparent}
}
% \usecolortheme[named=OliveGreen]{structure}
\setbeamertemplate{navigation symbols}{} 
\setbeamertemplate{blocks}[rounded][shadow=true] 

\newif\ifBCITCourse
\BCITCoursetrue
\BCITCoursefalse
\newif\ifWhichCourse
\WhichCoursetrue
% \WhichCoursefalse
\ifBCITCourse
\ifWhichCourse
\newcommand{\CourseName}{Statistics for Food Technology}
\newcommand{\CourseNumber}{MATH 1441}
\newcommand{\CourseInst}{BCIT}
\else
\newcommand{\CourseName}{Calculus for Geomatics}
\newcommand{\CourseNumber}{MATH 1511}
\newcommand{\CourseInst}{BCIT}
\fi
\else
\newcommand{\CourseName}{Philosophy and Literature}
\newcommand{\CourseNumber}{PHIL 375}
\newcommand{\CourseInst}{UBC}
\fi

\title{Poststructuralism and Skepticism}
\subtitle{{\CourseNumber}, {\CourseInst}}

\author{\CourseName}

\date{June 20, 2017}

\begin{document}

\begin{frame}
  \titlepage
\end{frame}

\begin{frame}
  \frametitle{iClicker Question}
Choose from the following options. This item will not be graded.
\begin{block}{iClicker Question}
The X argument: someone whose life is guided by usual practices is in
direct danger of becoming an X. Who is X?
\end{block}
\begin{description}
\item[A\hspace{.2in}$\blacktriangleright$] Oblomov
\item[B\hspace{.2in}$\blacktriangleright$] Eichmann
\item[C\hspace{.2in}$\blacktriangleright$] The Big Lebowski
\item[D\hspace{.2in}$\blacktriangleright$] Charles Duhigg
\end{description}
\end{frame}

\begin{frame}
  \frametitle{iClicker Question}
Choose from the following options. This item will be graded.
\begin{block}{iClicker Question}
What is the relevant pair of contrast for Odo Marquard's essay?
\end{block}
\begin{description}
\item[A\hspace{.2in}$\blacktriangleright$] accidents and narrative
\item[B\hspace{.2in}$\blacktriangleright$] accidents and coincidence
\item[C\hspace{.2in}$\blacktriangleright$] accidents and apriority
\item[D\hspace{.2in}$\blacktriangleright$] accidents and choices
\end{description}
\end{frame}

\begin{frame}
  \frametitle{The Necessity of the Accidental}
  Marquard argues that an element of the accidental is necessary for
  being human: 
  \begin{block}{Odo Marquard}
    To get rid of what is accidental [Hegel] would mean {\ldots} to
    get rid of philosophers {\ldots} without philosophers there would
    be no philosophy {\ldots} one would philosophy, in the name of
    philosophy, of philosophy (109)
  \end{block}
Later on, Marquard restates that one would ``rid man, in the name of
man, of man.''
\end{frame}

\begin{frame}
  \frametitle{Four Sections}
  \begin{enumerate}
  \item the program of making man absolute
  \item unavoidability of usual practices
  \item human beings are always more accidents than choice
  \item human freedom depends on the separation of powers
  \end{enumerate}
\end{frame}

\begin{frame}
  \frametitle{What Is the Accidental}
  The accidental emerges from an interesting property of the world as
  phenomenon (lifeworld, 121): that things could have been otherwise
  (Leszek Ko{\l{}}akowski: ``quod tamen potest esse aliter''). The
  accidental is what is neither impossible nor necessary. Distinguish
  between the
  \begin{itemize}
  \item arbitrarily accidental
  \item fatefully accidental (Nina Lugovskaya's strabismus)
  \end{itemize}
Si contingens, unde necessarium et vice versa. Again, we are
confronted with what looks like an irreducible dualism. Sartre,
following Kant: human beings are, without exception, not their
accidents, but only and completely their choice (contra ``moral
luck''). 
\end{frame}

\begin{frame}
  \frametitle{Absolutizing Ideologies}
  \begin{itemize}
  \item religion
  \item Descartes' methodical doubt: in dubito contra traditionem
    (when in doubt, against tradition)
  \item German idealism
  \item Marxism
  \item Habermasian discourse ethics
  \end{itemize}
Note how Marquard takes Gadamer's side in the debate over tradition;
and Popper's side in the debate over dogmatism/determinism/historicism
(open society). 
\end{frame}

\begin{frame}
  \frametitle{Marquard and Popper}
  Here are some similarities between Marquard and Popper:
  \begin{itemize}
  \item both were converted from adherence to abolutist ideologies
    (Popper from Marxism; Marquard from National Socialism)
  \item both assign an important role to creativity and the accidental
    in epistemology
  \item both vigorously criticize historicism (the idea that there is
    an intention in history beyond the intention of individuals which
    determines history)
  \item both emphasize the fallibility of knowledge
  \end{itemize}
\end{frame}

\begin{frame}
  \frametitle{Questions}
  \begin{itemize}
  \item ``human life is too short for executing the program of making
    man absolute'' (115)---compare and contrast the solution of German
    idealism/Marxism for this problem and Marquard's
  \item what is the difference between an epistemological skeptic like
    Descartes and an anti-dogmatic skeptic like
    Marquard/Popper/Ko{\l{}}akowski?
  \item note Marquard's unease with environmentalism: environmentalism
    has as one of its premises the ``eternal significance'' of human
    action, and our existing usual practices are often in conflict
    with this premise
  \item what is the Eichmann argument and how does Marquard address
    it?
  \end{itemize}
\end{frame}

\begin{frame}
  \frametitle{Questions}
  \begin{itemize}
  \item ``critique is, above all, conflict between usual
    practices'' (118)---what echoes of Kant do you hear here, and also
    a few lines later when Marquard rips into ``perfectionistic
    deontological demands''
  \item what kind of narrativist is Marquard? notice the difference
    between Aristotelian narratives (where form is imposed on content)
    and Marquardian narratives (where the contingent is necessary)
  \item note also similarities with Foucault: the significance of the
    contingent (versus Hume's necessity in the genealogy of morality,
    for example)
  \item note also similarities with Taylor: Marquard's emphasis on
    usual practices and tradition, which only make re-evaluation
    meaningful
  \end{itemize}
\end{frame}

\begin{frame}
  \frametitle{Questions}
  \begin{itemize}
  \item both Taylor and Marquard have a cautiously positive
    relationship with art and religion as human responses to mitigate
    arbitrariness
  \item skepticism (``Zweifel''): isosthenes diaphonia (in our course,
    for example, between the hermeneutic method and the scientific
    method)
  \item note Marquard's resistance to Mill's idea that there must be
    one principle of morality: ``there is never only one such power at
    work, but always a number of them'' (also for convictions,
    traditions, stories, souls, gods, points of orientation,
    freedoms---Marquard is a professing polytheist)
  \item finally, a nod to Jacques Derrida: ``what makes a human being
    free is not zero determination---the absence of all
    determinants---or the superior force of a single determinant, but
    a superabundance of determinants'' (124)
  \end{itemize}
\end{frame}

\begin{frame}
  \frametitle{What Is Enlightenment?}
  Human reason, says Marquard again with a bit of cheek directed at
  Immanuel Kant, is \alert{the abandonment of the effort to remain
    stupid}.
\end{frame}

\begin{frame}
  \frametitle{iClicker Question}
Choose from the following options. This item will be graded.
\begin{block}{iClicker Question}
What is Kolakowski's thesis statement?
\end{block}
\begin{description}
\item[A\hspace{.2in}$\blacktriangleright$] No explanatory method exists in the study of cultural history
\item[B\hspace{.2in}$\blacktriangleright$] facts can be abolished by a supreme synthesis
\item[C\hspace{.2in}$\blacktriangleright$] History would have turned out the same if certain important people (Descartes, Nietzsche) had died in infancy
\item[D\hspace{.2in}$\blacktriangleright$] Western civilization must decline as a matter of historical necessity
\end{description}
\end{frame}

\begin{frame}
  \frametitle{iClicker Question}
Choose from the following options. This item will be graded.
\begin{block}{iClicker Question}
According to Kolakowski, given the past, the future is
\end{block}
\begin{description}
\item[A\hspace{.2in}$\blacktriangleright$] unpredictable
\item[B\hspace{.2in}$\blacktriangleright$] directed at a specific end
\item[C\hspace{.2in}$\blacktriangleright$] predictable in its major movements
\item[D\hspace{.2in}$\blacktriangleright$] predictable in detail
\end{description}
\end{frame}

\begin{frame}
  \frametitle{iClicker Question}
Choose from the following options. This item will be graded.
\begin{block}{iClicker Question}
``The whole world is a work of art'' -- who is quoted saying this in Solnit?
\end{block}
\begin{description}
\item[A\hspace{.2in}$\blacktriangleright$] Virginia Woolf
\item[B\hspace{.2in}$\blacktriangleright$] Jean-Paul Sartre
\item[C\hspace{.2in}$\blacktriangleright$] Andrew Warhol
\item[D\hspace{.2in}$\blacktriangleright$] Immanuel Kant
\end{description}
\end{frame}

\begin{frame}
  \frametitle{iClicker Question}
Choose from the following options. This item will be graded.
\begin{block}{iClicker Question}
Why did Charlie Musselwhite stop drinking himself to death?
\end{block}
\begin{description}
\item[A\hspace{.2in}$\blacktriangleright$] because of a religious
  conversion experience
\item[B\hspace{.2in}$\blacktriangleright$] because he heard of a
  not-yet two-year old being saved from a well
\item[C\hspace{.2in}$\blacktriangleright$] because Aristotle
  appeared to him in a dream
\item[D\hspace{.2in}$\blacktriangleright$] because he was shipwrecked
  and had no access to alcohol
\end{description}
\end{frame}

\end{document}
