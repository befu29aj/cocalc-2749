% kafka.tex

\documentclass[11pt]{article}

\usepackage{october}

\usepackage{lineno}

% \hyphenpenalty=10000

% \hbadness=10000

\begin{document}

\onehalfspacing

\title{Moral Necessity and\linebreak{}the Implementation of Modernity}
% \title{Moral Necessity and the Implementation of Modernity}

\author{blind review}

\date{}

\maketitle

\section{Introduction}
\label{section:introduction}

There are two competing views on moral responsibility. One view gives
priority to \emph{conscience}, an essentially (or existentially) human
feature calling individuals to define themselves and perform actions
in keeping with a particular way in which human beings understand
themselves apart from or in addition to what it means to be human in
scientific or rational ways. Even though it is tempting to attribute
this view more broadly to continental philosophy rather than analytic
philosophy, I have selected a continental and an analytic philosopher
to represent it: Martin Heidegger and P.F. Strawson.

Another view embeds moral responsibility in the way in which we more
generally view decision-making and motivating behaviour. Moral agents
have distinctly moral ends (such as happiness for sentient beings or
human flourishing) to which the performance of morally motivated acts
is the means. Metaphysical questions as to what is or what is not
moral often retreat into the background for these accounts of moral
responsibility, whereas questions of rationality are foregrounded. A
moral agent must to some extent be epistemologically empowered in the
sense that her internal deliberations, intentions, choices, actions,
and the resulting outcomes are suitably aligned.

Even though it is tempting to attribute this view more broadly to
analytic philosophy (see, for example, the prominence of
utilitarian ethics in analytic philosophy), my view is that there
is a strong contingent in continental philosophy taking up this
viewpoint as well. Its approach, however, is often critical of the
possibility that human nature or the human condition is
meaningfully consistent with the requirements of morality. Some
continental thinkers follow the mainstream of analytic philosophy
that \qeins{reactive attitudes remain within the bounds of reason}
\scite{2}{russell92}{302}, but add the element that human beings
are not in the least equipped to meet standard rationality
requirements.

Some continental thinkers have thought through the implications of
what it means that on the one hand humans consider themselves morally
responsible and on the other hand scientific and technological
progress has put them in an unprecedented situation of epistemological
malaise with respect to understanding themselves. I have selected
Franz Kafka and Michel Foucault as representatives of this view;
neither of them analytic philosophers (one of them not a philosopher
at all), but it is my intention to show how their position can be
rendered intelligible in the terms of analytic moral philosophy and
what the consequences are for moral philosophy in general. 

\section{Implementation of Modernity}
\label{section:implementation}

In his article \qeins{Freedom and Resentment,} Strawson embarks on a
mission in moral philosophy that runs parallel to J{\"u}rgen
Habermas's idea of detranscendentalizing in the philosophy of
rationality. Immanuel Kant had given both the philosophy of
rationality and the philosophy of morality strongly transcendental
credentials, and both Strawson and Habermas are committed to saving
the enlightenment project from what Strawson calls
\qnull{over-intellectualizing.}

Strawson's argument proceeds as follows. There are optimists and
pessimists about the role of determinism for moral responsibility. The
optimists (compatibilists) do not consider pronouncements of whether
the world is or is not deterministic a threat to moral responsibility.
If anything, determinism supports the efficacy of making moral
distinctions because it helps to make concepts such as deliberation,
intention, and planning intelligible.

The problem, as the pessimists (libertarians, skeptics) point out,
is that moral distinctions would then fail to be merit-based, as
effective as they may be. Morality would devolve into
\qeins{intellectual understanding, management, treatment, and control}
\scite{2}{strawson08}{18}. For the pessimists, this devolution
constitutes a failure to give a coherent account of what is meant by
morality. Kafka and Foucault, who are Strawsonian pessimists of sorts,
consider this devolution to be representative for the implementation
of modernity, but more about this in a moment. Strawson's solution to
the pessimist's challenge is to detranscendentalize morality. The
pessimists have to give up on their stilted metaphysics; the optimists
have to give up on the moral picture \qeins{painted in a style
  appropriate to a situation envisaged as wholly dominated by
  objectivity of attitude. The only operative notions invoked in this
  picture are such as those of policy, treatment, control}
\scite{2}{strawson08}{22}.

It is central to Strawson's argument that moral judgments are not
primarily objective but rather participant attitudes. Humans have a
natural commitment to ordinary interpersonal attitudes. \qeins{This
  commitment is part of the general framework of human life, not
  something that can come up for review} \scite{3}{strawson08}{14}.
According to Strawson, we cannot take seriously the thought that a
high-level theoretical question such as whether or not determinism is
true would lead to the decay of the personal reactive attitudes.
Jonathan Bennett summarizes,

\begin{quote}
  if we try to imagine our lives without reactive feelings we find
  ourselves {\ldots} confronted by bleak desolation. We cannot be
  obliged to give up something whose loss would gravely worsen the
  human condition, and so reactive feelings cannot be made
  impermissible by any facts. \scite{3}{bennett80}{29}
\end{quote}

In keeping with a cultural trend in the philosophy of the mid-20th
century, Strawson's account of moral responsibility is
anti-metaphysical (consider, for example, the skepticism towards
metaphysical questions in and around the Vienna Circle). Moral
responsibility can be preserved from the threat of an invasive
scientific anthropology: at the expense of metaphysics and in favour
of, as Habermas calls it, a situated view of humanity. For Habermas,
transcendental rationality gives way to a rationality implemented in
communicative action that aspires towards an ideal speech situation.
For Strawson, the cold-hearted scientific approach of the optimist and
the metaphysical exuberance of the pessimist give way to a realistic
and embodied assessment of human psychology.

What violates this assessment is the idea that moral agents can fully
emancipate themselves from their participant attitudes and hold
objective attitudes instead. Strawson goes into some detail what the
dystopian consequences of a moral account would be that trades solely
in objective attitudes (examples may be the Kantian moral agent whose
moral sense is derivative of transcendental rationality; or the
utilitarian who performs a calculus to identify the optimal means to
achieving her moral aims).

\begin{quote}
  A sustained objectivity of inter-personal attitude, and the human
  isolation which that would entail, does not seem to be something of
  which human beings would be capable, even if some general truth were
  a theoretical ground for it. \scite{3}{strawson08}{12}
\end{quote}

Strawson goes on to describe the \qnull{abnormal egocentricity}
required to maintain that a metaphysical theory (such as determinism)
can lead to the collapse of moral structure in human behaviour and
motivation. Both human communities and individual personhood are based
on participant attitudes, therefore circumstances in which everyone is
morally incapacitated as a matter of metaphysical theory would lead to
a breakdown both on an individual and a social level.

My claim is that there is a sense in which Kafka and Foucault take up
this challenge and show in which ways it is not only possible that
Strawson's dystopia is implemented but that modernity has already
largely progressed to a state where both individuals and collectives
no longer meaningfully identify as moral agents.

Strawson has confident pronouncements about what is and what is not
essential to human nature.
% Blaise Pascal mused a few centuries earlier,
% \begin{quote}
%   what else does this craving, and this helplessness, proclaim but
%   that there was once in man a true happiness, of which all that now
%   remains is the empty print and trace? This he tries in vain to fill
%   with everything around him, seeking in things that are not there the
%   help he cannot find in those that are, though none can help, since
%   this infinite abyss can be filled only with an infinite and
%   immutable object; in other words by God himself.
%   \scite{3}{pascal66}{75}
% \end{quote}
One might claim that there will never be an end to poverty (Jesus
appears to do so in Matthew 26:11), tribalism, jealousy, sexual
harrassment in the workplace, or violence. Claims of this sort appeal
to human nature. For Kafka and Foucault, the problem is primarily not
that some human traits are genetically fixed or metaphysically
ordained, but that if anything the tendency is to underestimate the
ability of a human being to alienate herself from herself. Both Kafka
and Foucault are to some degree epigones of Sigmund Freud in this
instance, who was not chiefly interested in pathological
\emph{expression} of jealousy and violence, but in their pathological
\emph{suppression}. For Kafka and Foucault, the alienation of modern
humans from their attachment to moral responsibility becomes an issue
and a (grim) possibility.

In his book \emph{Being and Time}, Martin Heidegger argues that the
distance created by the existence of a human being and the fact that
this existence is an issue of concern for that human being becomes the
foundation for existence as \qnull{Dasein.} Dasein has a different
ontological structure than the existence of galaxies, chairs, the
Higgs boson, black swans, or unicorns. Moral responsibility is thus
immediately divorced from mere facts about external reality. In
$\S{}58$, Heidegger characterizes moral responsibility in terms of
conscience, a call by the self to the self towards the possibilities
of Dasein.

What Strawson and Heidegger have in common is the necessity of
de-intellec\-tualized morality for human life. For Strawson this is a
simple fact of human nature (one that may, for example, be open to
scientific inquiry). For Heidegger, it is a more fundamental condition
for existence as such. Guilt (one finds echoes of this view in Kafka's
work) is primarily an intransitive ontological condition rather than a
transitive consequence of the distinction between good and evil.
Transitivity is the grammatical feature of requiring an object, in
this case an object of guilt, the violation or transgression of a
rule.

In the following, I will discuss pushback against this insistence on
necessity for moral responsibility. In his account of justice in
\emph{A Treatise of Human Nature} (3.2.1), David Hume provides one of
the first instances of what Friedrich Nietzsche would later call a
\qnull{genealogy.} In this account, Hume relativizes the metaphysical
credentials of justice and characterizes it instead as organically
grown in the soil of social interaction among humans. It is critically
important to Hume, however, that justice arises from the social needs
of humans as a matter of necessity.

Foucault describes in his work, whether it is about the hermeneutics
of the subject, the penal system, schools, hospitals, prisons, or the
history of sexuality, that there is nothing necessary about the
particular forms that moral explanations take. They are rooted in
contingencies and are in the final analysis a downstream consequence
of the implementation of modernity (for Foucault, especially the
economic, institutional, and bio-political changes that modernity
brings about).

We now have a set of strong contrasts. The optimists versus the
pessimists of Strawson's argument; Strawson's psychological
assumptions about human nature versus Heidegger's existential
assumptions about the human condition; and finally, accounts of moral
responsibility that consider its most fundamental expressions to be
necessary to human life versus accounts which attribute to moral
responsibility ineradicable contingency.

In an obscure French essay about Kafka, Claude-Edmonde Magny makes a
compelling observation:

\begin{quote}
  One can find in Kafka's work a theory of responsibility, views
  on causality, finally a comprehensive interpretation of human
  destiny, all three sufficiently coherent and independent enough
  of their novelistic form to bear being transposed into purely
  intellectual terms.\footnote{The translation of the quote is
    from \scite{8}{blanchot95}{2}, the original quote is \qeins{on
      peut trouver dans l'{\oe}uvre de Kafka une th{\'e}orie de la
      responsabilit{\'e}, des vues sur la causalit{\'e}s, enfin
      une interpr{\'e}tation d'ensemble de la destin{\'e}e
      humaine, suffisamment coh{\'e}rentes toutes trois et assez
      ind{\'e}pendantes de leur forme romanesque pour supporter
      d'{\^e}tre transpos{\'e}es en termes purement intellectuels}
    (see \scite{8}{magny45}{}).}
\end{quote}

This paper, which is committed to the view that there is philosophical
insight in both the analytic and the continental tradition, has more
specifically two aims: to shed a critical light (i) on Strawson's
account of moral responsibility, especially its claims about
necessities of human nature; and (ii) on Kafka's ability to address in
literature philosophical claims that can be rendered analytically
intelligible. Kafka is a pessimist vis-{\`a}-vis the type of moral
agency envisioned by Strawson. I will try to detail this pessimism in
the next section.

\section{Kafka's Account of Moral Responsibility}
\label{section:kafka}

There is a view that Kafka's work is elusive with respect to
interpretation. Various schools of interpretive approaches to Kafka
have formed: Marxist, existentialist, psychoanalytic, and others. Most
commentators, however, claim some place on the sidelines of the
controversy between these schools and underline how open Kafka's texts
are to all sorts of interpretation. W.H. Auden, for example, states in
the first sentence of his essay on Kafka \qeins{The I Without a Self}
that Kafka is the master of a literary genre \qeins{about which a
  critic can say very little worth saying} \scite{2}{auden62}{159}.
For this view, it is the fluidity of Kafka's texts by which they
effortlessly move from being viewed from one perspective to being
viewed from an entirely different perspective that constitutes Kafka's
genius.

My claim, by contrast, is that there are themes that run through
Kafka's literary output which can be captured by interpretation,
translated into the terms of analytic philosophy (no doubt there will
be losses in this translation, losses nonetheless worth the rewards),
and used to manifest a definitive position that Kafka takes with
reference to topics such as moral responsibility, epistemological
access to causation, the status of metaphysical beliefs, and the
meaning of happiness.

There is an aporia in human life: we hold people morally accountable
for their actions and choices, yet we are also aware that human
decision-making proceeds in a rich and multi-layered field of causal
influences. Moral blame attenuates as a superficial investigation
yields to a deeper one. In the \emph{Nicomachean Ethics}, Aristotle
carefully distinguishes between different types of ignorance,
especially ignorance about one's true interests on the one hand and
ignorance about particular circumstances on the other hand. The former
ignorance characterizes vice, whereas the latter may annul it. A
writer like Kafka revels in the description of moral narratives that
lead these kinds of distinctions ad absurdum.

Kafka is not only a Strawsonian pessimist in the sense that the
metaphysics of modernity results in the devolution of moral coherence;
he is also an epistemological pessimist. He is not a skeptic in the
sense that nobody knows anything. Rather, the moral agents in Kafka's
work know and are deceived about a wide variety of things of great
importance to them, and as they investigate them (more often than not
in good faith), it becomes apparent that their experience of
epistemological failure in the past forestalls any epistemological
confidence in the future.

While moral agents may know things, they can never be sure of
things or trust their sensitivities. Knowing things does not
translate into knowing that you know them. This view is
incompatible with knowledge internalism, where $K(X)$, knowing
$X$, implies $K(K(X))$, knowing that one knows $X$. The moral
agent hardly ever has conscious access to the causal mechanisms
that inform morally responsible decision-making. What was yet
grotesque to Aristotle, that an agent should be ignorant of swaths
of relevant circumstances, particularly his or her own identity
(\qeins{now no one, unless mad, could be ignorant of all these
  circumstances together; nor yet, obviously, of the agent---for a
  man must know who he is himself,} \emph{Nicomachean Ethics},
1111a.1), turns into the cultural malady of an epoch.

\begin{quote}
  Everything of which we become conscious is arranged, simplified,
  schematized, interpreted through and through---the actual process of
  inner \qnull{perception,} the causal connection between thoughts,
  feelings, desires, between subject and object, are absolutely hidden
  from us (Nietzsche in \emph{The Will to Power}, section 477).
\end{quote}

I will use Nietzsche's and Foucault's idea of genealogy to show that
it is constitutive of considering oneself a moral agent that the
origins of moral agency remain masked for the moral agent. An
unmasking of a genealogical concept such as moral agency or
responsibility results in its collapse. In the context of Hume's
account of justice, genealogy denotes the kind of explanation pointing
to the origins of a social practice of which it is essential that they
themselves are not used as reasons to follow the practice. The core of
the practice is somehow constituted by a certain forgetfulness toward
its history. The forgetfulness is at the root of lending the practice
intrinsic rather than instrumental value. The intrinsic value becomes
detached from the original usefulness of the practice. It is a value
which experiences a threat to its reflective stability, and possibly a
breakdown, when its historical origins are uncovered.

Hume considers the emergence of justice from this genealogical process
to be a necessary and positive development---Nietzsche and Foucault,
by contrast, develop a critical theory of genealogy and strip it of
all elements of necessity. They give genealogical accounts of morality
and truth (Nietzsche), knowledge and sex/gender (Foucault), meaning to
undermine the intrinsic power of these concepts to motivate behaviour.

Deception is then not derivative of epistemological clarity but the
other way around: there is at first a deception needed to get moral
responsibility off the ground, namely that the diffuse and contingent
origins of moral agency are concealed from the moral agent; only then
can the moral agent come to some clarity of what her alternatives are
and how these may play out in the future. The agent then makes a moral
choice (such as Kafka's K. leaving Frieda to spend time with the
family of Barnabas in \emph{The Castle}), which is then vitiated by
future contingencies and future diffusions.

At the end, there is the existential guilt that Martin Buber has
identified in Kafka's texts (see \scite{8}{buber60}{}): a guilt which
is purely formal and lacks all content, for while humans live under
the burden of moral culpability, they cannot legitimately be made
responsible for anything in particular. Kafka expresses this
repeatedly in the \emph{Letter to His Father} when he identifies a
\qeins{boundless sense of guilt} \scite{2}{kafka66}{68} while
insisting that both he and his father are \qeins{entirely blameless}
(loc.\ cit., 4) and declaring the \qeins{guiltlessness of us both}
(loc.\ cit., 100).

Adam Smith, Charles Darwin, and Karl Marx consider an invisible hand
at work in history. Despite its diffusion (which, more positively, can
be cast as its essentially democratic core) and its lack of conscious
intention on the part of an author or creator it contributes towards
progress: in capitalism, the invisible hand creates welfare; in
evolutionary theory, it creates and multiplies life; and in Marxism,
it will not fail to bring about revolution and the victory of the
proletariat.

Nietzsche, Freud, and Foucault provide a more pessimistic picture of
the invisible hand. Foucault, for example, describes an intricate
network of micro-dominations which concatenate to produce technologies
of power that victimize without perpetrator. Sigmund Freud's invisible
hand operates during an individual's ontogenesis (rather than Darwin's
phylogenesis) in early childhood, where the origins and explanations
for these operations are wiped out by amnesia, which makes the history
of human agents inaccessible to them. The invisibility of this hand is
everything but benign, leading to widespread psychological pathology.

Nietzsche produces an account of concepts---concepts which to most of
us sound metaphysically and eternally established---that locates their
origins in the contingencies of human history: concepts such as
morality, love, justice, knowledge, and truth. Bernard Williams
describes Nietzsche's idea of genealogy as follows:

\begin{quotex}
  A genealogy is a narrative that tries to explain a cultural
  phenomenon by describing a way in which it came about {\ldots} our
  ethical ideas are a complex deposit of many different traditions and
  social forces, and they have themselves been shaped by
  self-conscious representations of that history. However, the impact
  of these historical processes is to some extent concealed by the
  ways in which their product thinks of itself.
  \scite{3}{williams04}{28}
\end{quotex}

Foucault takes Nietzsche's ideas further and explains:

\begin{quotex}
  However, if the genealogist refuses to extend his faith in
  metaphysics, if he listens to history, he finds that there is
  \qzwei{something altogether different} behind things: not a timeless
  and essential secret, but the secrets that they have no essence or
  that their essence was fabricated in a piecemeal fashion from alien
  forms. \scite{3}{foucault77}{142}
\end{quotex}

Kafka anticipates many of Foucault's conclusions in his
literature. The enlightenment project of characterizing human life
as guided by reason and by the transparency of self and nature to
the third-person gaze of a human mind has failed. In his texts,
Kafka carries on a tradition that is discernible already in Fyodor
Dostoyevsky's \emph{Notes from the Underground}:

\begin{quotex}
  Before your eyes the object vanishes, the reasons evaporate, the
  culprit is not to be found, the offence becomes not an offence
  but a \emph{fatum}, something like a toothache, for which no one
  is to blame {\ldots} you haven't found the primary
  cause.\footnote{Compare also {\'E}mile Zola's character La
    Maheude in \emph{Germinal} exclaiming, \qeins{Mais on
      r{\'e}fl{\'e}chit, n'est-ce pas? On s'aper\c{c}oit qu'au
      bout du compte ce n'est la faute de personne {\ldots} non,
      non, ce n'est pas ta faute, c'est la faute de tout le
      monde,} page 497; and Martin Heidegger's dictum in
    \emph{Sein und Zeit,} \qeins{In der Allt{\"a}glichkeit des
      Daseins wird das meiste durch das, von dem wir sagen
      m{\"u}ssen, keiner war es,} section 27. Both of these quotes
    unfold their meaning specifically in the context of
    modernity.} \scite{3}{dosto04}{18}
\end{quotex}

Transitivity gives way to intransitivity. Objects vanish, as in Martin
Heidegger's fundamental ontology hermeneutics becomes operative on
\qnull{being} without a text to serve as an object for the
hermeneutics. Kafka expresses this idea in literature by recording no
particular work to do for the officials in \emph{The Castle} and no
object for Joseph K.'s guilt in \emph{The Trial}. There is also no
sovereign chooser of moral value as there is in existentialism.
Perpetrator-subjects vanish in favour of bureaucratic concealment, as
in Foucault's modern penal system. Again, Kafka reflects this in
\emph{The Castle} and \emph{The Trial.} Finally, moral responsibility
collapses because there is for it no epistemological access to what
the relevant causal relationships are: \qeins{no one is to blame
  {\ldots} you haven't found the primary cause.}

In Kafka's \emph{Metamorphosis}, Gregor \qeins{suffers in person from
  [the] evil consequences [of his employment], which he can no longer
  trace back to the original causes} \scite{2}{kafka95}{83}. In
\emph{The Castle}, K., \qeins{in order to obtain pardon, first had to
  establish guilt, and that's precisely what they denied him at the
  offices} \scite{2}{kafka98b}{214}. In the short story \qeins{On the
  Tram,} Kafka gives expression to the epistemological crisis that
precipitates a moral crisis:

\begin{quotex}
  I stand on the end platform of the tram and am completely unsure of
  my footing in this world, in this town, in my family. Not even
  casually could I indicate any claims that I might rightly advance in
  any direction. I have not even any defence to offer for standing on
  this platform, holding on to this strap, letting myself be carried
  along by this tram, nor for the people who give way to the tram or
  walk quietly along or stand gazing into shop windows. Nobody asks me
  to put up a defence, indeed, but that is irrelevant.
\end{quotex}

In another short story called \qeins{Resolutions,} Kafka concludes:

\begin{quotex}
  So perhaps the best resource is to meet everything passively, to
  make yourself an inert mass, and, if you feel that you are being
  carried away, not to let yourself be lured into taking a single
  unnecessary step, to stare at others with the eyes of an animal, to
  feel no compunction, in short, with your own hand to throttle down
  whatever ghostly life remains in you, that is, to enlarge the final
  peace of the graveyard and let nothing survive save that.
\end{quotex}

\section{Conclusion}
\label{section:conclusion}

Moral decisions that human beings make in relationship with each other
are comparable to the decisions a driver has to make when steering a
fast vehicle without brakes in the fog. Kafka is not a
nihilist---decisions matter. The driver in the fog makes consequential
decisions, but the driver does not have enough information to make
these decisions so that they become appropriate objects of moral
evaluation.

Kafka had particular philosophical ideas which he used to give shape
to the literature that he created. He was skeptical about accounts of
moral responsibility produced by the enlightenment, such as Kant's
critique of practical reason or J.S. Mill's consequentialism, because
these moral theories make epistemological assumptions that are not
borne out in human life. They fail by the standards of descriptive
moral theory, which for both Kafka and Foucault is importantly primary
to prescriptive moral theory. Descriptive moral theory is interested
in the ways in which our behaviour is morally motivated and how these
motivations interact with institutions, interests of power, the
economy, contingencies of culture, and the management of bodies.

For Strawson, one way in which philosophers (both optimists and
pessimists) have erred is that they have subordinated prescriptive
moral theory to a particular type of description, the metaphysical
theory of determinism. On the one hand, if Strawson is correct and
it is in principle not possible to undermine fundamental moral
commitments by high-level descriptions, then Kafka's and
Foucault's project has failed. On the other hand, if Kafka and
Foucault succeed in providing a persuasive account (the one a
literary account, the other a philosophical account) describing
just the kind of human isolation and moral alienation that
Strawson considers inconceivable, then a weakness in Strawson's
argument is revealed.

\bibliographystyle{ChicagoReedweb}

\bibliography{bib-3660}

\end{document}

%configuration={"latex_command":"pdflatex -synctex=1 -interact=nonstopmode 'kafka.tex'"}